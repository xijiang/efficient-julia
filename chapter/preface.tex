\chapter*{Preface}
\addcontentsline{toc}{chapter}{Preface}

This book used a template from \url{https://github.com/sisl/tufte_algorithms_book}.
It initially ran on a Ubuntu system.
I setup a dedicated virtual Ubuntu machine for this template.
Besides, I also installed below packages:

git,
emacs,
vim,
texlive-extra-utils,
texlive-latex-extra,
python3-pip,
pdf2svg,
texlive-realscripts,
texlive-xetex,
texlive-fonts-extra,
texlive-science,
biblatex,
texlive-latex-extra,
rubber,
texlive-bibtex-extra,
texlive-pictures

Python package pygments was installed with pip.
Package pygments\_julia was installed with command

pip install --upgrade git+https://github.com/sisl/pygments-julia\#egg=pygments\_julia

This book template provides a starting point upon which authors may freely build to generate their own textbook entirely in \LaTeX.
We used this setup for Algorithms for Optimization, and have continued to refine it for a new textbook on decision making under uncertainty.
The template allows for the direct compilation of a print-ready PDF, including support for figures, examples, and exercises.

This template is intended for textbook authors.
Use of the template assumes prior expose to \LaTeX.
This template uses the \texttt{tufte-book} class to provide wide side margins.

Fundamental to our textbooks are the algorithms, which are all implemented in the Julia programming language.
We have found the language to be ideal for specifying algorithms in human readable form.
Algorithms can be typeset in algorithm blocks, tested, and used to generate figures.
We use pygments and pythontex, including a custom pygments lexer and style.

I thank the template authors.
Now I am starting to use it to write my own book,
The efficient use of Julia (for animal breeders).

\vspace{5ex}
\noindent\textsc{Xijiang Yu}\\
\AA s, Norway\\
Oct. 6, 2022
\vfill
\noindent Ancillary material is available on the template's webpage:\\
\noindent\url{https://github.com/sisl/textbook_template}
